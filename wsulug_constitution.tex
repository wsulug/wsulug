\documentclass{article}
\addtolength{\oddsidemargin}{-.875in}
\addtolength{\evensidemargin}{-.875in}
\addtolength{\textwidth}{1.75in}
\addtolength{\topmargin}{-.875in}
\addtolength{\textheight}{1.75in}

\begin{document}
\title{WSULUG Constitution}
\author{Wayne State University Linux User Group}
\date{September 2011}
\maketitle

\section{Goals and Purposes}
The purpose of the Wayne State University Linux User Group is to support and promote the use of Linux, UNIX and other Open Source technologies at Wayne State University and the surrounding community.

\begin{enumerate}
\item The group will hold regular installfests, installing Linux for any person who chooses to use Linux.
\item The group will hold technical presentations, to teach members about Linux and software running on Linux.  The presenters will be WSULUG members, or guest speakers.  These presentations will be open to the public. Membership is not a requisite to attend.
\item The group will hold informal discussions to discuss Linux and Open Source, and to assist other members who may have problems or questions.
\item The group will maintain a web site with resources to aid those interested in Linux, and information regarding WSULUG and meetings.
\item The group will maintain a mailing list, where members can hold discussions, request help, discuss WSULUG business, or anything else that may be related to the Linux and Open Source.
\end{enumerate}

\section{Requirements for Membership}
Full membership shall be open to all registered students of Wayne State University, who are interested in Linux and Open Source.  Wayne State University faculty and staff, and people in the surrounding community that are not Wayne State University students, may be affiliate members. There shall be no membership fee.

\section{Officers}
\subsection{List of Officers}
\begin{description}
\item{\textbf{President}}
This individual is the official liaison to the University, and is responsible for the group, its activities, presiding over meetings, and the delegation of tasks to be carried out on behalf of the group.
\item{\textbf{Vice President}}
The Vice President shall assist the president with his/her duties, and preside over the organization during the president's absence.
\item{\textbf{Secretary}}
The Secretary shall record, keep, and make available copies of the meeting minutes, and dispatch correspondence as directed by the president.
\item{\textbf{Treasurer}}
The Treasurer shall receive, dispatch, and keep accurate records of all financial matters regarding the Linux User Group.
\item{\textbf{Webmaster}}
The Webmaster shall maintain the group web site and mailing list.  All other officers may make modifications, to the web site and mailing list, if necessary.
\item{\textbf{Other Officers}}
Addition officers may be created to service the organization. The duties and titles of these offices will be created and establishment in the organization bylaws.
\end{description}
\subsection{Eligibility}
Any full member, who has attended at least 33\% of all meetings during the current academic year, may be nominated for an office.
\subsection{Nominations and Elections}
Any full member may nominate any other member, or themselves to be added to the ballot.  If the nomination is seconded, and the nominee approves, then the individual shall be added to the ballot. Nominees will be given a maximum of 5 minutes to present a speech to the general membership explaining their reasons behind running for office, qualifications, goals, and all other applicable information. When all nominees have completed their speeches, any other member may voice his/her support for a particular candidate with a maximum of 2 minutes to present his/her speech.  At least one week shall pass before a vote is taken on the officers. The election of officers shall be done through anonymous paper ballots, during a regularly scheduled meeting, or a special meeting called for the purpose of electing officers.  Any full member, including nominees, who have attended at least 33\% of meetings prior to the election day may vote.  The nominees receiving the greatest number of votes will immediately take the office. In the case of a tie, the quorum of the membership will vote again, taking only those candidates involved in the tie into consideration. If the group is ever without a president for any reason, the highest ranking officer will act as the president until an election is held for a new president.  The order of succession will be Vice President, Secretary, Treasurer, Webmaster. Officers will be nominated at the third group meeting, after the beginning of the Fall semester.  At least one week will pass after the nominees are selected, for the election to take place.  Officers will hold their respective offices until an election is held during the next academic year.

\subsection{Replacing Officers}
Any officer or member that fails to uphold his or her post may be removed from office by a vote or at least 67\% by a quorum.  Any full member may propose a recall, during a regularly scheduled meeting, or a special meeting called for the purpose of recalling an officer.  If the proposal is seconded, then a recall election will be held.  Any individual who wishes to speak for or against the recall shall be given the opportunity to do so at this time. At least one week shall pass before a recall vote is taken.
If an officer is recalled, steps down, or is unable to complete his/her term in office for any other reason, then the group must start the process or electing a new officer at the next group meeting.  This will be done according to the rules of electing a new officer.


\section{Meetings}
\subsection{Quorum}
A quorum shall consist of at least 33\% of the group members, including at least 40\% of the officers.  At least one of the officers present must be the President or Vice President, provided those positions are not vacant.
\subsection{Method of Conducting}
Meeting shall be run according to the Democratic Rules of Order.
\subsection{Special Meetings}
Any member may call for an emergency meeting or special event through a request on the group's mailing list.  If 33\% of the group agree, with e-mailed replies to the list, then the meeting or event will be held. The Secretary shall be responsible for informing membership of emergency or special meetings. Members will be informed of emergency meeting through both, the WSULUG web site and the mailing list.
\subsection{Day and Time}
The elected officers shall decide the time, date, and location of meetings. This information will be distributed by e-mail and posted to the web site. The decision to hold a special meeting may be made by the elected officers when necessary. Meetings may consist of virtual meetings or real meetings. Meetings may be cancelled or rescheduled if the elected officers agree that a meeting may result in undue academic stress upon the group members.

\section{Process of Amending Constitution}
Any full member may propose an amendment to the constitution, during a regularly scheduled meeting, or a special meeting called for the purpose of amending the constitution. If the amendment is seconded, then the amendment shall go to a vote. Any individual who wishes to speak for or against the amendment shall be given the opportunity to do so at this time. The officers will decide the amount of time given to speakers based on the amount of time available. This time will be no less than 2 minutes, and no greater than 5 minutes. At least one week shall pass before a vote is taken on the amendment. If at least 67\% of the full membership in the quorum vote to support the amendment, this constitution shall be amended.
\end{document}
